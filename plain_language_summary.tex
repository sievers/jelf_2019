\documentclass[letterpaper,11pt,preprint]{aastex}
\usepackage{graphics,graphicx}
\usepackage{natbib}
\usepackage{color}
\citestyle{aas}

\setlength{\textwidth}{6.5in} \setlength{\textheight}{9in}
\setlength{\topmargin}{-0.0625in} \setlength{\oddsidemargin}{0in}
\setlength{\evensidemargin}{0in} \setlength{\headheight}{0in}
\setlength{\headsep}{0in} \setlength{\hoffset}{0in}
\setlength{\voffset}{0in}

\makeatletter
\newcommand{\mbe}{{\rm MBE}}
\renewcommand{\section}{\@startsection%                                                                                                                                                                    
{section}{1}{0mm}{-\baselineskip}%                                                                                                                                                                         
{0.5\baselineskip}{\normalfont\Large\bfseries}}%                                                                                                                                                           
\makeatother

%%%%%%%%%%%%%%%%%%%%%%%%%%%%%                                                                                                                                                                              
%%%%% Start of document %%%%%                                                                                                                                                                              
%%%%%%%%%%%%%%%%%%%%%%%%%%%%%                                                                                                                                                                              

\begin{document}
\pagestyle{plain}

\section{Plain Language Summary}

Building on the success of the Canadian Hydrogen Intensity Mapping
Experiment (CHIME), we propose to build a new transportable back-end
for next-generation radio telescopes that can be deployed to sites
across the planet avoiding human-generated interference.  Arrays of
radio telescopes consist of collecting elements (often, but not
always, dishes) with sensitive receivers that amplify the incoming
radio waves.  These {\textit{front-ends}} then send their signals to a
central location where the {\textit{back-end}} processes the incoming
signals and combines them together to produce scientifically useful
data.  When measured by the number of incoming data streams processed
simultaneously, the McGill back-end (\mbe) proposed here will be the
second largest in the world.

The first deployment will be to the Hydrogen Intensity and Real-time
Analysis eXperiment (HIRAX), and array under development in South
Africa.  Consisting of up to 1024 6m-diameter dishes, HIRAX will study
dark energy and pinpoint thousands of fast radio bursts, amongst many
other things.  

During HIRAX operations, we will develop deployable
low-frequency antennas and use them to search for radio-quiet sites in
northern Canada, where we can search for the birth of the first
stars.  This will pave the way for the next deployment of the \mbe.  
\end{document}
