\documentclass[letterpaper,11pt,preprint]{aastex}
\usepackage{graphics,graphicx}
\usepackage{natbib}
\usepackage{color}
\citestyle{aas}

\setlength{\textwidth}{6.5in} \setlength{\textheight}{9in}
\setlength{\topmargin}{-0.0625in} \setlength{\oddsidemargin}{0in}
\setlength{\evensidemargin}{0in} \setlength{\headheight}{0in}
\setlength{\headsep}{0in} \setlength{\hoffset}{0in}
\setlength{\voffset}{0in}

\makeatletter
\newcommand{\mbe}{{\rm MBE}}
\renewcommand{\section}{\@startsection%                                                                                                                                                                    
{section}{1}{0mm}{-\baselineskip}%                                                                                                                                                                         
{0.5\baselineskip}{\normalfont\Large\bfseries}}%                                                                                                                                                           
\makeatother

%%%%%%%%%%%%%%%%%%%%%%%%%%%%%                                                                                                                                                                              
%%%%% Start of document %%%%%                                                                                                                                                                              
%%%%%%%%%%%%%%%%%%%%%%%%%%%%%                                                                                                                                                                              

\begin{document}
\pagestyle{plain}

% 1500 characters
% Provide a short summary in plain language of the proposed project:
% what is being researched, how it is being done and why it is
% important. Focus on the expected impact and benefits to Canada, beyond
% academic accomplishments. This summary will not be used in the review
% process. Should the project be funded, it may be used in the CFI’s
% communications products and website.

\section{Plain Language Summary}

We present a multi-faceted proposal to build a novel, remotely
deployable signal processor---the McGill ``back-end'' (\mbe)---that
will be a game-changing addition to radio astronomy. The \mbe\ will be
built in a new radio instrumentation lab at McGill. This lab will also
build antennas to search for radio-quiet areas in northern Canada that
will serve as potential sites for the \mbe\ working in concert with
future radio telescope arrays.

The \mbe\ is a portable and powerful signal processing core. Radio
telescope arrays generate enormous data volumes, thus presenting a
daunting transport challenge. The \mbe\ is unique in that the system can
be moved physically to the telescopes, thus eliminating the burden of
transferring data remotely. The \mbe\ will first serve as the heart of
the Hydrogen Intensity and Real-time Analysis eXperiment (HIRAX),
which will be sited in South Africa. With the \mbe\ in place, HIRAX will
produce a wealth of data for elucidating dark energy and transient
phenomena.

The new McGill radio instrumentation lab will also develop antennas
for use in northern Canada. Northern Quebec offers extensive fiber
optic and road networks, while being sparsely populated.  This rare
combination yields a unique geographic advantage for radio astronomy.
We will use our antennas to search for areas that are radio-quiet yet
have good infrastructure; these sites can potentially host future,
next-generation radio telescope arrays, which will be empowered by the
\mbe.

% Building on the success of the Canadian Hydrogen Intensity Mapping
% Experiment (CHIME), we propose to build a new transportable back-end
% for next-generation radio telescopes that can be deployed to sites
% across the planet avoiding human-generated interference.  Arrays of
% radio telescopes consist of collecting elements (often, but not
% always, dishes) with sensitive receivers that amplify the incoming
% radio waves.  These {\textit{front-ends}} then send their signals to a
% central location where the {\textit{back-end}} processes the incoming
% signals and combines them together to produce scientifically useful
% data.  When measured by the number of incoming data streams processed
% simultaneously, the McGill back-end (\mbe) proposed here will be the
% second largest in the world.
% 
% The first deployment will be to the Hydrogen Intensity and Real-time
% Analysis eXperiment (HIRAX), and array under development in South
% Africa.  Consisting of up to 1024 6m-diameter dishes, HIRAX will study
% dark energy and pinpoint thousands of fast radio bursts, amongst many
% other things.  
% 
% During HIRAX operations, we will develop deployable
% low-frequency antennas and use them to search for radio-quiet sites in
% northern Canada, where we can search for the birth of the first
% stars.  This will pave the way for the next deployment of the \mbe.
\end{document}
