\documentclass[letterpaper,11pt,preprint]{aastex}
\usepackage{graphics,graphicx}
\usepackage{natbib}
\usepackage{color}
\citestyle{aas}

\setlength{\textwidth}{6.5in} \setlength{\textheight}{9in}
\setlength{\topmargin}{-0.0625in} \setlength{\oddsidemargin}{0in}
\setlength{\evensidemargin}{0in} \setlength{\headheight}{0in}
\setlength{\headsep}{0in} \setlength{\hoffset}{0in}
\setlength{\voffset}{0in}

% \makeatletter
% \newcommand{\mbe}{{\rm MBE}}
% \renewcommand{\section}{\@startsection%
% {section}{1}{0mm}{-\baselineskip}%
% {0.5\baselineskip}{\normalfont\Large\bfseries}}%
% \makeatother

% HCC's SPACE SAVING MAGIC FOR THE WINZ
\setlength{\parskip}{0.07in}
\setlength{\bibsep}{0.05in}

\makeatletter
\newcommand{\mbe}{{\rm MBE}}
\def\subsize{\@setsize\subsize{12pt}\xipt\@xipt}
\def\section{\@startsection {section}{1}{\z@}{1.0ex plus 
1ex minus .2ex}{.2ex plus .2ex}{\large\bf}}
\def\subsection{\@startsection {subsection}{2}{\z@}{.2ex 
plus 1ex} {.2ex plus .2ex}{\subsize\bf}}
\makeatother

%%%%%%%%%%%%%%%%%%%%%%%%%%%%%                                                                                                                                                                              
%%%%% Start of document %%%%%                                                                                                                                                                              
%%%%%%%%%%%%%%%%%%%%%%%%%%%%%                                                                                                                                                                              

%\usepackage{fontspec}
\usepackage{fancyhdr}
\pagestyle{fancy}
\headheight=12pt
\voffset -0.3in
\textheight=9.0in
\headsep=18pt
\lhead{McGill University}
\chead{Sievers \& Chiang}
\rhead{Project \# 38558}
\renewcommand{\headrulewidth}{0pt}
\renewcommand*{\bibfont}{\tiny}
%\usepackage{multicol}
%\usepackage{etoolbox}
%\patchcmd{\thebibliography}{\section*{\refname}}
%    {\begin{multicols}{2}[\section*{\refname}]}{}{}
%\patchcmd{\endthebibliography}{\endlist}{\endlist\end{multicols}}{}{}

\begin{document}

%\setmainfont{Times}

%\pagestyle{plain}
\pagenumbering{gobble}
\singlespace

%\section{Introduction}

%Building on the success of the Canadian Hydrogen Intensity Mapping
%Experiment (CHIME), we propose to build a new transportable back-end
%for next-generation radio telescopes that can be deployed to
%radio-quiet sites across the planet.  

% Traditionally, the back-ends were
%so expensive that it made sense to spend a lot of money on building
%large dishes and keeping the back-end fixed.  At its heart, the
%back-end is just a large computer, so Moore's law is having a dramatic
%impact on the traditional radio-telescope design.  This fact, plus switching to
%commodity parts, has lead to the back-end cost per incoming signal
%dropping by orders of magnitude.  In turn, for many science cases it
%no longer makes sense to build expensive traditional radio
%telescopes.  Instead, building very many cheap front-end elements
%coupled with a powerful back-end becomes a natural design.  

%{\setstretch{1.0}

As observational cosmologists, we are trying to understand some of the
most basic properties of the universe.  What is the nature of dark
energy?  Is the tension between different measurements of the Hubble
constant pointing to new fundamental physics?  When did the first
objects in the universe light up?  Were they stars or something else?
What did the universe look like before these first stars formed, the
so-called ``cosmic dark ages''?  Can we use fast radio bursts (FRBs)
as a new probe of cosmology?  In the last few years, radio astronomy
has become one of our most promising avenues for answering these
questions.  Driven by Moore's law, a new generation of radio arrays
consists of relatively inexpensive telescopes connected to a
``back-end'' of powerful digital signal processing electronics.  

Building on the successful technology developed at McGill for the
Candaian Hydrogen Intensity Mapping Experiment (CHIME
\citet{Bandura16}), we propose to build the McGill back-end (\mbe),
which will handle the inputs from up to 1,024 radio telescopes and
will process 3.25 terabits per second.  By many measures, the
\mbe\ will be the second largest of its kind in the world, surpassed
only by CHIME itself.  The \mbe\ will be unique in that it will be
portable and hence can be used to power multiple different radio
telescopes over its useful life. In the work covered by this proposal,
we will build and integrate the \mbe\ at McGill.  It will first be
deployed, along with a diesel rotary uninterruptible power supply
(DRUPS) as part of the Hydrogen Intensity and Real-time Analysis
eXperiment (HIRAX \citet{Newburgh16}), the southern hemisphere
complement to CHIME.
% where we
%hope to answer many of our questions about the universe. 
Sievers and Chiang, are co-PI and hardware lead respectively of the
international HIRAX collaboration.  Along with our students and
postdocs, we will use HIRAX and the \mbe\ to answer fundamental questions
about the universe.
%As co-PI and
%hardware lead, respectively, of the HIRAX collaboration, proposers
%Sievers and Chiang, along with their students and postdocs, will have
%full access to all HIRAX data. 
Concurrently, we will investigate sparsely populated areas of northern
Canada for promising sites for future \mbe\ deployments after HIRAX
(which will take 4-5 years to survey the southern sky),
which will address new questions in cosmology.

\section{Research/Technology Development}

Hydrogen atoms naturally emit radio waves at a frequency of 1.4 GHz or
a wavelength of 21 cm.  As the universe expands, it stretches out
photons or ``redshifts'' them, driving them to lower frequencies.
This simple fact means that observing hydrogen at lower frequencies
shows us the universe when it was younger and more compact.  A radio
telescope that observes over a range of frequencies then, in a single
observation, naturally maps out the universe over a large fraction of
its life.  We can select {\it{which}} periods of the universe's life
we look at by selecting the corresponding frequencies.  Flexible
electronics like the \mbe\ can then be used to study many different
phases in the evolution of the universe simply by switching which
frequencies we feed into it.  This gives us the freedom to keep our
focus on the most pressing areas of cosmology in a way that was simply
not feasible in the past.  In this section, we will describe the core
science goals of the \mbe's first deployment at the heart of HIRAX,
followed by how modern radio telescopes work and the role the \mbe\ 
fills.  Finally, we will describe how northern Canada's rare
combination of good infrastrucure in sparsely populate areas makes it
a promising site for future \mbe\ deployments looking at the very young
universe.


\subsection{HIRAX and the \mbe\ as  Probes of Dark Energy}

When Albert Einstein worked out the equations of general relativity,
he saw that the simplest solution did not allow for a static
universe. Faced with this seeming unpleasantness, Einstein added a
term to his equations, called the cosmological constant, that he
thought would allow the universe to be infinitely old.  This term
arose from fixing the value of an integration constant to counter the
gravity from matter, and corresponds to giving a non-zero energy
density to empty space.  When presented with Edwin Hubble's evidence
for an expanding universe, Einstein dropped the cosmological constant,
calling it the ``biggest blunder'' of his life.  

There the matter stood for many decades, with the cosmological
constant being ignored.  From dimensional arguments, its natural value
is a Planck mass per Planck length cubed, or about $5\times 10^{96}$
kilograms per cubic metre.  To get a sense of scale for just how huge
that density is, it corresponds to compressing all the matter in the
observable universe into a region the size of the nucleus of a single
atom.  Since the natural value is so large, it only made sense that
the cosmological constant be precisely zero.  However, as cosmologists
probed further and further into the universe, things began not quite
adding up.  In the 80's and 90's, studies of how distant galaxies
clumped together indicated that large-scale structures weren't growing
as expected.  In a universe with only matter in it ({\textit{e.g.}
  \citet{Park94}}.  Then in the late '90's, two separate groups
announced that distant supernovae were fainter than expected when
compared to nearby ones \citep{Riess98,Perlmutter99} in work that
would lead to the 2011 Nobel prize in physics.  Their data were
inconsistent with a matter-only universe, but agreed with a universe
in which expansion accelerated due to a cosmological constant.
Finally, as measurements of the cosmic microwave background (CMB)
transformed cosmology in the 2000's, we began to get percent-level
accuracy measurements of the vacuum energy contribution, with the
latest constraints from the Planck satellite indicating that today,
vacuum energy makes up $68.89\pm0.56$\% of the mass-energy in the
universe\footnote{while early CMB measurements needed an external
  measurement of the Hubble constant to make these statements, the
  improving measurements of the gravitational lensing of the CMB mean
  this is no longer true}\citep{Planck2018Params}.  As members of the
Atacama Cosmology Telescope (Sievers), South Pole Telescope (Chiang),
and Planck (Chiang) collaboration, the proposers have co-authored
dozens of papers over the last decade measuring the properties of the
universe and sharpening constraints on the vacuum energy.  

The disagreement between the natural ($10^{96}$ kg/m$^3$) and observed
(equivalent to about one atom per cubic metre) values of the vacuum
energy is ``probably the worst theoretical prediction in the history
of physics! \citep{Hobson06}'' As such, there is no particularly good
reason to think that the simplest model, that of a cosmological
constant, is correct.  Cosmologists have packaged their ignorance by
calling the vacuum contribution ``dark energy'', with the cosmological
constant but one of many possible models \citep{Ratra88,Steinhardt99}.
It {\textit{may}} turn out that a cosmological constant is a good
description, but it may well not.  If cosmologists were to find that
the vacuum energy changes with time, it would radically reshape our
view of the universe and point to profoundly new fundamental physics.
So far, the data we have in-hand are broadly consistent with constant
dark energy, within rather large errors.  However, it is of the utmost
importance to probe that by pushing to higher sensitivity and earlier
times in the universe, where different models for the time evolution
might be more apparent.  The need for improved dark energy constraints
over a broad range of time is particularly pressing because
values of the Hubble constant derived from low and high redshift data
now disagree at the 5-$\sigma$ level.  If this discrepancy is found to
be real and not due to systematic errors, it could point to evolving
dark energy (see {\it{e.g.}} \citet{Freedman2017} and many others).
The tension has now even captured the popular
imagination\footnote{https://www.scientificamerican.com/article/have-we-mismeasured-the-universe/
  is one of several recent overviews}, heightening the urgency of
improved measurements.


%One way that dark energy is often
%parameterized is in terms of an {\textit{equation of state}}, where
%$w=\mathrm{P}/\rho$, where $\mathrm{P}$ is the pressure associated
%with dark energy, an $\rho$ is the energy density.  For a pure
%cosmological constant, $w=-1$.  A further generalization is to let $w$
%be a function of the size of the universe, $w(a)=w_0+(1-a)w_a$.  In
%this parameterization (others are possible), $a$ is the scale factor, the ratio of the size
%of the universe at a given point in time relative to the size today,
%and is the inverse of one plus the redshift $z$, $a=\frac{1}{1+z}$.  
%By definition, $a$ today is equal to one, and so measuring $w_0$ tells
%us about the modern day behaviour of dark energy, and $w_a$ tells us
%something about how that behaviour changes with time.  In these terms,
%the best current constraints are $w_0=-0.961 \pm 0.077$ while
%$w_a=-0.28^{+0.31}_{-0.27}$ \citep{Planck2018Params}.  So, while the
%evidence that dark energy behaves close to a cosmological constant
%today is moderately good, our limits on its evolution are very weak.
%This parameterization also has been used as a way of quickly comparing
%different experiments via the dark energy figure of merit (FoM), which
%is just one over the area of the 1-$\sigma$ contours in the $w_0,w_a$
%plane.  

Fortunately, nature has given us ways to map out the story of dark
energy, which affects how the universe has expanded as a
function of time.  The detection using supernovae relied on
the fact that astronomers think they know their true brightness and
can use them as {\textit{standard candles}}.  With a standard candle,
by comparing the observed brightness of an object with its known true
luminosity, one can work out its distance.  So, when supernovae
appeared fainter than their measured redshfits would have predicted,
dark energy was a natural explanation.  In addition to standard
candles, we can also use {\textit{standard rulers}}, a population of
things whose intrinsic size we know.  Then by comparing the observed
angular size we see to the known true size, we can again work out a
distance.  Nature has helpfully filled the universe with such standard
rulers, called baryon acoustic oscillations (BAOs).  

% BAOs arise from
%the first few hundred thousand years of the universe, when the density
%was nearly uniform everywhere and sound waves are the natural
%description.  The sound waves were driven by gravity and by pressure
%from free electrons dragging photons with them via Thomson
%scattering.  About 400,000 years after the big bang, the free
%electrons and protons very quickly combined to form neutral hydrogen
%atoms.  With no more free electrons, the sound waves stopped, and
%density perturbations began to grow simply via gravity, with their
%relative amplitudes frozen in.  We see these amplitudes today
%imprinted in the distribution of galaxies, with galaxy positions
%correlated on $\sim$500 million light-year scales today.  These ripples are
%called {\textit{baryon acoustic oscillations}} or BAOs.

BAOs arise from the same physics in the first 400,000 years after the
big bang that gives rise to the cosmic microwave background (CMB)
power spectrum.  Because the size scale of BAOs is set during the
young universe, and the CMB power spectrum is exquisitely well
measured \citep{PlanckSpectra2015,Louis2017,Henning2018} cosmologists
understand very well how to tie the basic physics of the universe to
the BAO scale.  The BAO scale turns out to be large ($\sim$500 million
light years today), so their physics remains essentially linear, and
they simply scale with the size of the universe.  Therefore they
provide an almost ideal population of standard rulers. By observing
their angular size, we can map the history of dark energy.

The first detection of BAOs \citep{Eisenstein05} was made mapping
individual galaxies, but the very large scale of BAOs gives us another
tool, that of {\textit{intensity mapping}}.  Given the large size,
there can be hundreds of thousands of galaxies in any BAO volume, and
so one can take a very low-resolution image of the sky that averages
over large numbers of them.  The 21cm hydrogen line is an especially
attractive tool for doing this.  Because hydrogen is the most abundant
element in the universe, the 21 cm line is relatively strong, and
unlikely to be confused by other radio spectral lines.  So, a radio
telescope that can map the sky at low resolution over a broad
frequency range can make a full 3-dimensional image of the BAOs, and
therefore trace out the history of dark energy.

HIRAX will be an array designed to measure the BAO spectrum over a
large area (up to 15,000 square degrees, or about 40\% of the sky) and
a large fraction of the history of the universe (from 2 to 6 billion
years after the big bang, or redshift from 2.5 to 0.8).  We need this
large volume and time range to measure enough BAO volumes that we have
good statistics, and to see how dark energy is affecting the growth of
the universe.  The hydrogen emission is faint ($100 \mu \mathrm{K}$),
so we need a very sensitive telescope.  The signal is on large
(degree) scales, though, so we do not need a very high resolution
telescope.  These facts set the HIRAX design of $\sim$ 1000
close-packed 6m-diameter dishes.  The HIRAX collaboration will build
the experiment in stages, starting with a 128-dish pathfinder,
followed by 512, and finally a full array of 1,024 dishes.  Even with
only 512 dishes, HIRAX will provide improved constraints on the
evolution of dark energy in comparison to current measurements from
all of cosmology (see Figure \ref{fig:hirax_de}).  Measured by the
dark energy task force figure of merit\footnote{The figure of merit is
  defined relative to a model where the dark energy equation of state
  has a constant term $w_0$ and and term that evolves with the size
  scale of the universe $w_a$ where the size scale $a\equiv(1/1+z)$.}
(FoM), HIRAX-512 will have an FoM value of 103, compared to the
current state of the art of 95 (from CMB+BAO+supernovae,
\citet{Planck2018Params}).  Crucially, HIRAX is more sensitive at
earlier times than current measurements, so if the onset of dark
energy domination occurred earlier than in the standard cosmological
model, HIRAX is uniquely positioned to observe this effect.  The final
1024-element HIRAX array will have a projected FoM of 285, which is
comparable to or better than next-generation surveys such as eBOSS in
the 100-200 range \citep{Zhao2016}.


%Hydrogen atoms naturally emit radio waves at a frequency of 1.4 GHz,
%but the photons emitted long ago get stretched out by the expansion of
%the universe.  This stretching means we see these photons today at
%lower frequencies, with the observed frequency telling how far back in
%time we are looking.  So, we can use sensitive radio arrays to look at
%low frequencies and see the universe as it was billions of years ago.
%Key science questions we can address with hydrogen observations
%include: what is the nature of dark energy, that is accelerating the
%expansion of the universe?  When did the first objects in the universe
%light up?  Were they stars or something else? What did the universe
%look like before these first stars formed, the so-called ``cosmic dark
%ages''?  What are fast radio bursts, mysterious objects that flash
%brightly for a thousandth of a second?  A new generation of radio
%telescopes is poised to answer these fundamental questions about our
%universe.  Key to this new generation is the harnessing of Moore's law
%to combine the signal from ever larger numbers of individual detectors
%together enabling them to work as one coherent telescope array.  The
%actual radio telescopes and amplifiers change depending on the science
%questions, but the ``back-end'' that processes and combines the
%signals from the array looks very similar for all of these questions.

%Arrays of radio telescopes
%consist of collecting elements (often, but not always, dishes) with
%sensitive receivers that amplify the incoming radio waves.  These
%{\textit{front-ends}} then send their signals to a central location where the
%{\textit{back-end}} processes the incoming signals and combines them together to
%produce scientifically useful data.  

%We propose to build the McGill back-end (\mbe), a deployable back-end
%capable of processing 1,024 input streams with a combined data rate of
%3.25 terabits per second.  We will use the \mbe to try to answer some
%of these fundamental questions about the universe.  Based on
%technology successully developed for the Canadian Hydrogen Intensity
%Maping Experiment (CHIME), its initial deployment will be to South
%Africa where it will form the heart of the southern analog to CHIME,
%the Hydrogen Intensity and Real-time Analysis eXperiment (HIRAX).
%During HIRAX's four year observing run, we will search sites in
%northern Canada for locations free from human-generated interference,
%laying the groundwork for the next deployment of the \mbe.  Measured by the number
%of input streams processed together, this back-end will be the second
%largest in the world, second only to CHIME itself.  Due to the huge
%volume of data processed, we need to bring the \mbe\ to remote sites,
%rather than build the extensive infrastructure to bring terabits per
%second of data back to Montreal.    


% This
%transformation has led to the a new model in which many inexpensive
%telescopes are piped into increasingly sophisticated digital%
%``back-ends'' that process vast amounts of data.  These back-ends are
%becoming increasingly flexible, capable of handling signal from a wide
%range of radio telescopes operating over wide ranges of frequency.
%This enables a new mode where the same back-end can be used with many
%different telescopes answering a wide range of basic questions about
%the universe.  However, practical considerations coupled with the
%questions being asked strongly influences the ideal sites to place
%radio telescopes.  Moore's law has now made it possible to make
%cutting-edge back-ends portable.  
%We propose to build a deployable
%back-end capable of processing 1,024 input streams with a combined
%data rate of 3.25 terabits per second.  Based on technology
%successully developed for the Canadian Hydrogen Intensity Maping
%Experiment (CHIME), its initial deployment will be to South Africa
%where it will form the heart of the southern analog to CHIME, the
%Hydrogen Intensity and Real-time Analysis eXperiment (HIRAX).  During
%HIRAX's four year observing run, we will search sites in northern
%Canada for sites free from human-generated interference, laying the
%groundwork for the next deployment.  Measured by the number of input
%streams processed together, this back-end will be the second largest
%in the world, second only to CHIME itself.

%\section{Research/Technology Development}%

%The Hydrogen Intensity and Real-time Analysis eXperiment
%\citep{Newburgh16} is a planned radio aray that will be sited in the 
%South African Karoo desert, on the future Square Kilometre Array (SKA)
%site.  HIRAX will consist of a large number of 6 metre diameter dishes
%(up to 1,024), in a compact layout.  Two of the central questions
%HIRAX will address are what is the nature of dark energy?  And what
%are fast radio bursts?  With its southern hemisphere location, HIRAX
%will also be perhaps the premier instrument for discovering new
%pulsars outside of the plane of the Milky Way.   Since the first
%deployment of the McGill back-end (\mbe) will be with HIRAX, we
%describe its science case first, followed by the essentials of how a
%radio telescope works and how the \mbe\ fits into HIRAX.  We then
%describe the current status and deployment plans for HIRAX, and
%discuss its complementarity it to CHIME.
%%the Canadian Hydrogen Intensity Mapping Experiment (CHIME).  

%\subsection{HIRAX as a Probe of Dark Energy}

%When Albert Einstein worked out the equations of general relativity,
%he saw that the simplest solution did not allow for a static
%universe. Faced with this seeming unpleasantness, Einstein added a
%term to his equations, called the cosmological constant, that he
%thought would allow the universe to be infinitely old.  This term
%arose from fixing the value of an integration constant to counter the
%gravity from matter, and corresponds to giving a non-zero energy
%density to empty space.  When presented with Edwin Hubble's evidence
%for an expanding universe, Einstein dropped the cosmological constant,
%calling it the ``biggest blunder'' of his life.  

%There the matter stood for many decades, with the cosmological
%constant being ignored.  From dimensional arguments, its natural value
%is a Planck mass per Planck length cubed, or about $5\times 10^{96}$
%kilograms per cubic metre.  To get a sense of scale for just how huge
%that density is, it corresponds to compressing all the matter in the
%observable universe into a region the size of the nucleus of a single
%atom.  Since the natural value is so large, it only made sense that
%the cosmological constant be precisely zero.  However, as cosmologists
%probed further and further into the universe, things began not quite
%adding up.  In the 80's and 90's, studies of how distant galaxies
%clumped together indicated that large-scale structures weren't growing
%as expected.  In a universe with only matter in it ({\textit{e.g.} \citet{Park94}}.  Then in 
%the late '90's, two separate groups announced that distant supernovae
%were fainter than expected when compared to nearby ones
%\citep{Riess98,Perlmutter99} in work that would lead to the 2011 Nobel 
%prize in physics.  Their data were inconsistent with a matter-only
%universe, but agreed with a universe in which expansion accelerated
%due to a cosmological constant.  Finally, as measurements of the
%cosmic microwave background (CMB) transformed cosmology in the 2000's,
%we began to get percent-level accuracy measurements of the vacuum
%energy contribution, with the latest constraints from the Planck
%satellite indicating that today, vacuum energy makes up
%$68.89\pm0.56$\% of the mass-energy in the universe\footnote{while
%  early CMB measurements needed an external measurement of the Hubble
%  constant to make these statements, the improving measurements of the
%  gravitational lensing of the CMB mean this is no longer
%  true}\citep{Planck2018Params}. 

%The disagreement between the natural ($10^{96}$ kg/m$^3$) and observed
%(equivalent to about one atom per cubic metre) values of the vacuum
%energy is ``probably the worst theoretical prediction in the history of physics! \citep{Hobson06}''
%  As such, there is no particularly
%good reason to think that the simplest model, that of a cosmological
%constant, is correct.  Cosmologists have packaged their ignorance by
%calling the vacuum contribution ``dark energy'', with the cosmological
%constant but one of many possible models \citep{Ratra88,Steinhardt99}.
%It {\textit{may}} turn out that a cosmological
%constant is a good description, but it may well not.  If cosmologists
%were to find that the vacuum energy changes with time, it would
%radically reshape our view of the universe.  So far, the data we have
%in-hand are consistent with constant dark energy within rather large
%errors, but it is of the utmost importance to probe that by pushing to
%higher sensitivity and earlier times in the universe, where time
%evolution might be more apparent.  One way that dark energy is often
%parameterized is in terms of an {\textit{equation of state}}, where
%$w=\mathrm{P}/\rho$, where $\mathrm{P}$ is the pressure associated
%with dark energy, an $\rho$ is the energy density.  For a pure
%cosmological constant, $w=-1$.  A further generalization is to let $w$
%be a function of the size of the universe, $w(a)=w_0+(1-a)w_a$.  In
%this parameterization (others are possible), $a$ is the scale factor, the ratio of the size
%of the universe at a given point in time relative to the size today,
%and is the inverse of one plus the redshift $z$, $a=\frac{1}{1+z}$.  
%By definition, $a$ today is equal to one, and so measuring $w_0$ tells
%us about the modern day behaviour of dark energy, and $w_a$ tells us
%something about how that behaviour changes with time.  In these terms,
%the best current constraints are $w_0=-0.961 \pm 0.077$ while
%$w_a=-0.28^{+0.31}_{-0.27}$ \citep{Planck2018Params}.  So, while the
%evidence that dark energy behaves close to a cosmological constant
%today is moderately good, our limits on its evolution are very weak.
%This parameterization also has been used as a way of quickly comparing
%different experiments via the dark energy figure of merit (FoM), which
%is just one over the area of the 1-$\sigma$ contours in the $w_0,w_a$
%plane.  

%Fortunately, nature has given us ways to map out the story of dark
%energy.  Dark energy affects how the universe has expanded as a
%function of time.  So, if we map that expansion out, we can work out
%the nature of dark energy.  The detection using supernovae relied on
%the fact that astronomers think they know their true brightness and
%can use them as {\textit{standard candles}}.  With a standard candle,
%by comparing the observed brightness of an object with its known true
%luminosity, one can work out its distance.  So, when supernovae
%appeared fainter than their measured redshfits would have predicted,
%dark energy was a natural explanation.  In addition to standard
%candles, we can also use {\textit{standard rulers}}, a population of
%things whos intrinsic size we know.  Then by comparing the observed
%angular size we see to the known true size, we can again work out a
%distance.  Nature has helpfully filled the universe with such standard
%rulers, called baryon acoustic oscillations (BAOs).  BAOs arise from
%the first few hundred thousand years of the universe, when the density
%was nearly uniform everywhere and sound waves are the natural
%description.  The sound waves were driven by gravity and by pressure
%from free electrons dragging photons with them via Thomson
%scattering.  About 400,000 years after the big bang, the free
%electrons and protons very quickly combined to form neutral hydrogen
%atoms.  With no more free electrons, the sound waves stopped, and
%density perturbations began to grow simply via gravity, with their
%relative amplitudes frozen in.  We see these amplitudes today
%imprinted in the distribution of galaxies, with galaxy positions
%correlated on $\sim$500 million light-year scales today.  These ripples are
%called {\textit{baryon acoustic oscillations}} or BAOs.

%Because the size scale of BAOs is set during the young universe,
%cosmologists understand very well how to tie the basic physics of the
%universe to the BAO scale.  Because the BAO scale is so large, their
%size just scales with the expansion of the universe.  Therefore they
%provide an almost ideal population of standard rulers - we know how
%big they are physically given how much the universe has expanded, and
%so by observing their angular size, we can map the history of dark
%energy.  

%The first detection of BAOs \citep{Eisenstein05}
%was made with galaxies, but the very large scale of BAOs gives us
%another tool, that of {\textit{intensity mapping}}.  Given the large
%size, there can be hundreds of thousands of galaxies in any BAO
%volume, and so one can take a very low-resolution image of the sky
%that averages over large numbers of them.  The hyperfine emission line
%of atomic hydrogen is an especially attractive tool for doing this.
%Hydrogen atoms in the universe radiate at a rest frequency of 1.4 GHz
%(or a wavelength of 21 cm) due to the energy difference between the
%proton and electron sping being aligned and anti-aligned.  Because
%hydrogen is the most abundant element in the universe, this makes the
%21 cm line relatively strong, and unlikely to be confused by other
%radio spectral lines.  So, by making a low-resolution map of hydrogen
%emission, we can trace out the BAOs.  Furthermore, because the
%expansion of the universe stretches out the 21cm photons, by observing
%at longer wavelengths, we can see the universe when it was younger.  A
%radio telescope that can map the sky with over a broad frequency range
%then can make a full 3-dimensional image of the BAOs, and therefore
%trace out the history of dark energy.  

%HIRAX will be an array designed to measure the BAO spectrum over a
%large area (up to 15,000 square degrees, or about 40\% of the sky) and
%a large fraction of the history of the universe (from 2 to 6 billion
%years after the big bang, or redshift from 2.5 to 0.8).  We need this
%large volume and time range to measure enough BAO volumes that we have
%good statistics, and to see how dark energy is affecting the growth of
%the universe.  The hydrogen emission is faint ($100 \mu
%\mathrm{K}$), so we need a very sensitive telescope.  The signal is on
%large (degree) scales, though, so we do not need a very high
%resolution telescope.  These facts set the HIRAX design of $\sim$ 1000
%close-packed 6m-diameter dishes.  We plan to build HIRAX in stages,
%first with 128 dishes, then 512, and finally a full array of 1,024
%dishes.  With 512 dishes, HIRAX constrains the evolution of dark
%energy comparable to current measurements from all of cosmology (see
%Figure \ref{fig:hirax_de}, and at 1,024 elements is comparable to
%next-generation experiments.  


\begin{figure}[tbh]
  \includegraphics[height=2.5in]{hirax_bao_ps.pdf}
  \includegraphics[height=2.5in]{hirax_de_constraints.pdf}
\caption{\small Left: HIRAX measurement of BAOs as a function of
  spatial wavenumber.  Blue/yellow/green are for four years of 50\%
  efficiency observations with 128/512/1024 dishes.  The size of the
  ripples decreases to the right, and the ripples are the periodic
  structure frozen in at recombination, when neutral hydrogen atoms
  formed.  Right: HIRAX constraints on dark energy cast in the
  $w_0,w_a$ plane.  With 512 dishes, HIRAX alone will improve on the
  current state of cosmology, which has a figure of merit of 95 (from
  CMB+BAO+supernovae, \citet{Planck2018Params}).  With the full 1024
  dishes, HIRAX will be comparable to next-generation dark energy
  surveys, which typically have figures of merit of $\sim 200$.
  \label{fig:hirax_de}
}
\end{figure}

\subsection{HIRAX and the \mbe\ as a Probes of Fast Radio Bursts}
Fast radio bursts (FRBs) are some of the most enigmatic objects in the
universe.  The first one was discovered by \citet{Lorimer07} in 2007,
just a decade ago.  As far as we know, they flash once, brightly, for
a thousandth of a second, and are generally never seen again (only two
FRBs have been seen to repeat).  While we know virtually nothing about
what FRBs are, they almost certainly arise from well outside the Milky
Way.  While theories abound, what truly gives rise to these flashes
bright enough to be seen from across the universe is a mystery.  While
we expect several thousand visible FRBs per day, the small fields of
view of current telescopes means that since their discovery, only
around 60 have been discovered, so their properties are not well
understood. Sievers' research has become increasingly focussed on
FRBs, including how to find them \citep{Masui15}, possible origins of
FRBs \citep{Connor2016}, and how we can use them as a new probe of
cosmology \citep{Madhavacheril18}.

While HIRAX was designed to study dark energy, it is also a nearly
ideal instrument to find FRBs.  With 6m dishes operating between 400
and 800 MHz, the HIRAX field of view is $\sim 25-50$ square degrees.
The compact layout of HIRAX improves the efficiency of the FRB search
as well.  Modern algorithms have reduced by orders of magnitude the
computational load of searching a single beam for FRBs, to the point
that HIRAX will be able to search the $\sim$thousand beams that fill
its entire field of view.  Telescopes built for high resolution,
however, struggle to do this, and so they are likely to miss many
events.  HIRAX's high efficiency, along with the sensitivity from
having nearly 30,000 square metres of collecting area, means that we
expect to find nearly as many FRBs {\textit{every few days}} as have
been reported in total to-date.  While the rate in the HIRAX band is
still uncertain, with a rough value of 10-25 per day, CHIME has
recently reported FRB detections \citep{CHIME_ATEL,chime_frbs} ruling
out pessimistic estimates.

While a critical step to understanding what FRBs are is finding more,
an equally important step is to get accurate positions for them.  With
sufficiently accurate positions, extragalactic FRBs can be tied to a
host galaxy.  With a host galaxy, one can measure a reshift, and hence
a distance, to the FRB.  With a distance, one can finally infer an
energy scale for the bursts.  So far, despite extensive searches, only
two FRBs have been seen to repeat.  Knowing where to look, astronomers
have pinpointed the location of the first repeating FRB to a dwarf
galaxy roughly 2.5 billion light years away ($z=0.19$).  Hundreds of
bursts have now been seen from that repeater, so we fundamentally do
not know if it is even typical of the FRB population.  {\textit{No
    non-repeating FRB has yet to be localized.}}  When coupled with
outrigger stations the collaboration is building across southern
Africa (and for which funding has already been secured), HIRAX will
change the field.  With a total of several dozen inexpensive dishes at
a handful of sites working together with the core array powered by the
\mbe, HIRAX will be able to determine the positions of a large
fraction of non-repeating FRBs to a thirtieth the size of typical
distant galaxies (0.03 arcsecond accuracy for 15$\sigma$
detections\footnote{The search space for FRBs is so vast that with
  perfectly Gaussian statistics, 9$\sigma$ fluctuations will be a
  regular occurence.  Any non-idealities will drive that up, so
  15$\sigma$ is a conservative but realistic threshold for FRB
  detections.} in the core array).  Not only will HIRAX plus
outriggers be able to localize thousands of FRBs, it will be able to
say from where within galaxies FRBs arise.  If FRBs are found to
originate near the cores of galaxies, their progenitors are likely to
be very different than if they arise from the outskirts.  With
localizations, we can even start to think using FRBs as probes of
cosmology, since they exquisitely measure the electron column density
along the line of site. This was first pointed out in
\citet{McQuinn2014}, who showed that just 100 FRBs with positions
(which HIRAX could find in a week) could be used to locate the
``missing baryons''.  They can also be used as probes of fundamental
physics, including placing limits on the photon mass
\citep{bonetti2017} and testing Einstein's equivalence principle
\citep{frb_equivalence}.  Sievers and collaborators have also shown
that FRBs can also be used to probe the kinetic Sunyaev-Zeldovich
effect, breaking the velocity-optical depth degeneracy in
next-generation CMB observations \citep{Madhavacheril18}.

\subsection{Radio Telescopes and the \mbe}

An array of radio telescopes works by measuring the raw electric
fields from each individual telescope, and combining the signals from
every pair of telescopes.  This coherent combination of all the pairs
of telescopes lets the array work as a single whole, with
diffraction-limited resolution set not by the size of the individual
telescopes, but by the size of the array as a whole.  While individual
radio telescopes tend to have very poor resolution, arrays of radio
telescopes (often called interferometers) have given astronomers the
highest-resolution images ever taken.  For instance, the Event Horizon
Telescope, which combines telescopes across the planet to try to image
black holes, could resolve an orange on the moon\footnote{60
  micro-arcseconds resolution -
  https://eventhorizontelescope.org/building-larger-array}.

The way that interferometers combine the signals has a major impact on
the design, so we here lay out their basics.  First,
individual elements collect and amplify incoming radio waves.  These
elements might look like traditional radio telescopes, with large
parabolic reflectors (examples include JVLA, ALMA, MeerKAT, amongst
many others).  They might however consist of individual antennas
(PAPER, LWA), or multiple antennas hooked together with a first analog
combination stage (MWA, LOFAR), or even long cylinders with amplifiers
strung down the centre (CHIME, Ooty, UTMOST).  For modern digital
correlators, the next step is to digitally sample the electric fields,
and then use Fourier transform-based techniques to split the signals
into many narrow frequency channels.  At this point, all the frequency
channels from one antenna are together, but since we want to combine
all pairs of antennas together, we have to re-shuffle our data.  One
can think of the data as a 3-dimensional array, with time along one
axis, frequency along the second, and antenna number along the thirds.
This reshuffling, called the ``corner turn'' by radio astronomers,
amounts to take a matrix transpose in the frequency-antenna plane.
Once the data have been grouped by frequency, then every pair of
antennas are cross-correlated, which is mathematically equivalent to a
matrix multiply.  This design, where the data are split into frequency
channels, then cross-correlated, is called an FX correlator.  The part
that carries out the frequency splitting is usually referred to as the
F-engine, and the part that does the cross-correlation as the
X-engine.  The time-averaged cross correlation of a pair of antennas
(referred to as a baseline) is called a visiblility, and is the
fundamental output of a radio interferometer.

While the individual steps in an FX correlator are conceptually
simple, the challenge is handling the truly immense amounts of data.
For the 512-dish version of HIRAX, we will have 1024 signals coming
into the \mbe, each with 400 megahertz of bandwith.  Using 4-bit
samples, that works out to over 3.2 terabits of data per second.  That
is nearly 20 times the data rate of the entire CANARIE network (2017
average), which services 157 Canadian higher education institutions
and over a million
users\footnote{https://www.canarie.ca/about-us/facts/}.  The
\mbe\ will take the incoming radio-frequency (RF) signals, digitally
sample them, split them into component channels, and carry out the
corner turn using custom backplanes.  It will then pass the
corner-turned data to a graphics processing unit (GPU)-based
correlator that will actually calculate the visibilities.  The
digitizing/sampling boards and custom backplanes were developed over
many years by Matt Dobbs' group at McGill, and successfully power CHIME.
% Each crate
%with a custom backplane in the \mbe\ has 16 FPGA-based ICE boards
%(developed by Matt Dobbs' group at McGill).  Each board handles 16
%inputes, so a full crate works on 256 input RF streams.  The backplane
%in the crates carries out the corner turn for its 256 inputs, sending
%signals out over 10 gigabit ethernet cables.  For thousand
%element-scale arrays like HIRAX, Each GPU correlator node can take a
%10 Gb link from each \mbe\ crate, completing the corner turn, and the
%many terabits of data can be shuffled without the use of a single
%network switch.  
The \mbe\ will sit in a shielded, air-conditioned shipping container,
and all hardware will be installed, integrated, and tested at McGill
before being shipped to the HIRAX site.  The highly modular nature
also means that the \mbe\ can be expanded simply by adding more crates
as future funding permits.  The GPU correlator will be housed
separately.  Rapid developments in GPUs mean that the
cross-correlation part of back-ends have become obsolete on short
timescales.  For instance, NVidia has improved the on-paper
performance on 4-bit matrix multiplies, the core of what the GPUs do,
by a factor of $\sim$50 in about two years.  A single consumer-grade
board now has a theoretical performance of over 400 teraops for 4-bit
matrix multiplies.  So, the correlator one would design today is
radically different from the one CHIME deployed last year.
Converseley, the digitizing/channelizing/corner-turning operations
that the \mbe\ will carry out evolve much more slowly, and use much
less power (1.5 kW per crate).  By decoupling the F from the X part of
the back-end, the \mbe\ will remain a powerful tool for many years to
come, and can be connected to multiple telescopes and multiple GPU
engines over its useful life.  This explains why the \mbe\ can be
deployed to South Africa for HIRAX and then redeployed to northern
Canada with a new, by then much-cheaper and lower power, GPU engine.

\subsection{HIRAX Deployment Plan and Status}

HIRAX will be one of the largest radio telescopes in the world.
Hence, we are planning a staged rollout.  In 2017, we built an
8-element prototype at the Hartebeesthoek Radio Astronomy Observatory
(HartRAO) outside of Johannesburg, South Africa.  The prototype has
been invaluable in testing dishes and front end components, and in
gaining experience with back-end components.  The 8-element prototype
and its first-light data are shown in Figure \ref{fig:hirax8}.  It
uses one of the same McGill-developed boards that will go into the \mbe.
%  The next
%phase will be to build a second prototype on the Radio-frequency
%interference (RFI) protected site of the future Square Kilometre Array
%(SKA), which is also home to MeerKAT, the South African SKA
%pathfinder.  
The South African Radio Astronomy Observatories (SARAO) has agreed to
host HIRAX on SARAO-owned land about 15 km from the core MeerKAT site
(30.6885S,21.5689E), where the radio -frequency interference (RFI)
environment is excellent in the HIRAX 400-800 MHz band (see right
panel of Figure \ref{fig:hirax8}). SARAO is in the process of rezoning
the Swartfontein farm where HIRAX will be located, which will be
completed in Q3 2019, at which point we will be allowed to begin
construction on the final site.  We will start with a 128-element
array (HIRAX-128), with observations starting in 2020.  With 128
elements, we can save all the raw visibility data from the array, and
finalize and verify the data calibration/compression pipeline. After
many months of observations with HIRAX-128, we will start construction
of a 512-element version.  HIRAX-512 will be an extremely powerful
radio telescope in its own right, with limits on dark energy
comparable to our current knowledge from all cosmological data
combined.  The \mbe\ as proposed will support the full
HIRAX-512. After a year of observing with HIRAX-512, funding
permitting, we will begin on HIRAX-1024.  We will survey the entire
southern sky visible to HIRAX, which we expect will take approximately
4 years.  As part of our site agreement, SARAO will provide access to
power and data connections for HIRAX.  As part of the HIRAX
installation, the University of KwaZulu-Natal will purchase and
install a diesel rotary uninterruptible power supply (DRUPS)
sufficient to power the full array.  At the completion of HIRAX
operations, McGill will assume ownership of the DRUPS, and it and the
\mbe\ will return to Canada for our next observing programme (beyond
the scope of this proposal).

% However, that power envelope will be
%needed by SKA itself when it begins operations.  HIRAX will have
%finished its survey of the southern sky by the start of SKA, at which
%point the \mbe\ and the DRUPS will return to Canada where we will have
%developed our next observing programme.


%30 41 18.73S 21 34 07.98 
\begin{figure}[tbh]
  \includegraphics[height=1.4in]{hirax8.png}
  \includegraphics[height=1.6in]{first_fringe.png}
  \includegraphics[height=1.6in]{hirax_site_rfi.png} 
\caption{\small Left: HIRAX 8-element prototype using off-the-shelf
  Chinese-manufactured 6m dishes.  The back-end uses the same boards
  that we will use for the \mbe.  Center:  The first-light data from
  the prototype.  The vertical stripes are human-generated
  interference (predominantly UHF TV stations). This interference
  would severely hamper HIRAX's ability to see BAOs.  The ripples at
  the top of the plot are the interference pattern from a bright
  source crossing through the field of view and show that the
  prototype is functioning end-to-end.  Right:  The RFI spectrum from
  the final HIRAX site on RFI-protected land owned by SARAO.  Unlike
  the prototype site, where at least half the data are lost to RFI,
  there is no visible RFI at the sensitivity level of the test
  equipment in the HIRAX science band.
  \label{fig:hirax8}
}
\end{figure}


%\begin{figure}[tbh]
%  \includegraphics[height=2.5in]{hirax_site_v2_marked.png}
%  \includegraphics[height=2.5in]{hirax_site_rfi.png}
%\caption{\small Left: The location of the HIRAX site in the South
%  African Karoo desert.  The site is about 15 km away from the MeerKAT
%  core site, and 75 km from Carnarvon, the nearest town.  The site is
%  on RFI-protected land owned by SARAO.  Right: Daytime RFI spectra
%  measured from several locations on the HIRAX site.  At the
%  sensitivity of the test equipment, there is no detectable RFI in the
%  HIRAX 400-800 MHz band.  Cell phones are visible at 925-950 MHz,
%  military satellites appear in the 300 MHz range, and
%  intermittent RFI from airplanes is visible around 1.1 GHz.  Analog
%  filters will remove the out-of-band RFI at the telescope before
%  signals are sent to the back-end.
%  \label{fig:hirax_site}
%}
%\end{figure}

Much of the development work for HIRAX has been completed, with the
main challenge scaling to hundreds of dishes.  The technology for the
\mbe\ was developed for CHIME \citep{Bandura16}, and works in the
field.  Similarly, the GPU correlator \citep{Recnik15} 
works at scale in the field as well, with CHIME having discovered 13
FRBs \citep{chime_frbs}.  Custom feeds
have been designed and built, and we continue to iterate to improve
their noise performance.
%  The signals from the telescopes will be sent
%via optical fiber to the back-end, in a procedure called radio
%frequency over fiber (RFoF).  Prototype RFoF transmit and receive
%boards have been demonstrated in the field at HartRAO, and the first
%production-scale order has been received and is undergoing testing.
%The main outstanding hardware development is the construction of the
%6m dishes.  
We have used cheap ($\sim$800 USD) Chinese-made dishes at HartRAO, but
their optical design is not ideal for HIRAX (we require dishes with a
focal ratio of $f=0.25$, which shields the feeds and reduces crosstalk
between nearby antennas, key for a clean cosmological measurement),
and the manufacturing quality is rather poor.  Consequently, we have
been working with Canadian and South African partners to develop
suitable dishes.  The leading option is a composite dish with a reflective
wire mesh embedded in fiberglass.  Canadian HIRAX collaborator Gordon
Lacy (NRC) has built two half-scale prototypes, and has verified their
surface accuracy to have an RMS of 350$\mu$m.  A full scale
production-quality mold is nearing completion in South Africa, with a
measured surface RMS of 800$\mu$m, less than 0.3\% of a wavelength.
The half-scale prototype and production mold are shown in Figure
\ref{fig:hirax_dishes}. 
%  A second route is laser-cut
%metal dishes, with wire mesh attached to ribs. A full scale prototype
%has been completed, with a second ordered to compare repeatability.
%Samples of the dish prototypes are shown in Figure
%\ref{fig:hirax_dishes}.  We plan to verify the prototypes and make a
%final dish selection by the end of 2018.

\begin{figure}[tbh]
  \includegraphics[height=2.5in]{3m_dish.jpg}
%  \includegraphics[height=2.5in]{rebcon_dish.png}
  \includegraphics[height=2.5in]{mms_mold.jpg}
\caption{\small Left:  Canadian designed and built half-size composite
  HIRAX dish prototype (photo credit: Gordon Lacy, NRC).  For a few
  hundred dollar dish, the surface accuracy is outstanding, with an
  RMS of 350$\mu$m.  Right: production mold in South Africa, with a
  surface RMS of 800$\mu$m.
%Full-scale composite molds are currently being
%  manufactured in South Africa.  Right:  Full-sized metal prototype,
%  built in South Africa.  Surface accuracy and repeatability are still
%  being measured, but we expect them to meet HIRAX specifications.
  \label{fig:hirax_dishes}
}
\end{figure}


HIRAX has already received a total commitment of 20,000,000 South
African Rand (about \$2,000,000 canadian dollars, though exchange
rates are volatile) from the South African National Research
Foundation (NRF) and the University of KwaZulu-Natal for hardware
funding, along with significant funding for graduate students and
postdocs.  We are fully funded for the HIRAX-128, and with this
proposal plus expected future funding from the NRF would be at the
level needed for HIRAX-512.  That would utilize the full capacity of
the \mbe.  We stress that expanding the \mbe\ to handle HIRAX-1024
simply requires buying more ICE board crates and installing them in
the RF-shielded room.

%\subsection{Complementarity With CHIME} HIRAX follows on the groundbreaking
%work carried out by the CHIME team, while being complementary in many
%key ways.  First and foremost, the dark energy survey is extremely
%challenging.  Foreground emission from the Milky Way is $\sim$1000
%times brighter than the intensity mapping signal we are searching for.
%To succeed, both CHIME and HIRAX will need to exquisitely control
%systematic errors at that part-per-thousand level.  CHIME uses
%cylinders while HIRAX uses dishes, and so the systematics from the
%telescopes should be radically different.  If CHIME and HIRAX agree,
%it strengthens the case for both experiments that the systematics have
%been properly accounted for.  Due to its southern hemisphere location,
%there is virtually no sky in common between HIRAX and CHIME, and the
%results from both can be combined to improve dark energy constraints
%even more.  A fitting analogy is that CHIME and HIRAX will complement
%each other in the same way that the ground-based CMB experiments South
%Pole Telescope (SPT) and the Atacama Cosmology Telescope (ACT) have
%complemented each other, to the benefit of the field.  HIRAX's
%southern hemisphere location also gives it an edge in searching for
%new pulsars, since the pulsar disribution is centered on the galactic
%center, which is in the southern sky.  The HIRAX site looks
%significantly more quiet than the CHIME site in British Columbia,
%where the Canadian government has not taken the step of turning off TV
%stations for radio astronomy.  The nominal collecting area of HIRAX is
%significantly greater.  While CHIME has an impressive 8,000 square
%metres of collecting area, HIRAX-1024 will have just under 30,000
%square metres, a factor of 3 improvement.  This will lead to much
%improved sensitivity to FRBs and pulsars.  The design based on cheap
%dishes also makes building the FRB localization outriggers much
%simpler.  Ideally the outriggers and the main array have the same
%collecting elements, which means they can have the same sky coverage.
%Should CHIME push toward FRB localization, they will have to build
%more cyliners, at what is likely to be significantly greater cost than
%building more HIRAX 6m dishes.  

\subsection{RF Lab and Antenna Development}
The redshifted hydrogen line can be used in cosmology in many more
ways than just studies of dark energy.  It is amongst the most
powerful tools we have to look back in time at the birth of the first
stars.  For instance, if confirmed, the EDGES detection of a
stronger-than-expected absorption trough at 78 MHz from ``cosmic
dawn''\citep{Bowman2018} would radically reshape our view of the early
universe.  The maximum expected signal is set by simple phyics, and so
factor of $\sim$2 stronger signal they found would indicate we are
missing a major piece in the story of the universe, and possibly of
new fundamental physics, such as early dark matter-baryon interactions
\citep{Barkana2018}.  While we do not propose here for deployments of
the \mbe\ past the first one, we do wish to lay the groundwork for
future low-frequency work.  The challenges associated with RFI at
lower frequencies are even more intense, with FM radio in particular
blinding most telescopes to a critical time in the universe's history.
Site selection is a major factor, with places far from civilization
and sheltered from radio transmissions ideal.  With its sparse
population but good infrastructure, northern Canada is particularly
attractive for low-frequency work.  We have deployed small radio
antennas to remote locations, including possibly the most remote radio
telescope in the world \citep{PRIZM}.  It is located on Marion Island,
halfway between South Africa and Antarctica, 2000 km from the closest
permanent inhabitants.  This proposal will enable the development of a
radio lab that we will use to build and test more self-contained
antennas.  We will then carry out site tests at promising locations we
will undertake to identify.

To carry out this work, Cynthia Chiang will develop a radio lab at
McGill.  The lab will first and foremost have the test equipment
necessary to verify the performance and integration of the \mbe.  It
will also be used to build and test the test antenna systems to be
deployed in the Canadian north.  The lab will house the sensitive
benchtop RF test equipment (such as vector network analyzers) needed
for these roles, plus serve as home base to portable RF test equipment
that can verify system performance in the field.  


\section{Researchers}
PI Sievers has a long history in cosmology with a background
originally in studies of the CMB (using both inteferometers and
bolometric cameras), with work on galaxy clusters and parameter
estimation from gravitational waves, with 21cm cosmology now a central
part of his work.  He has co-authored over 100 scientific
publications, with over 10,000 citations and an $h$-index of 54
(Google Scholar).  One key relevant contribution was developing new
fast pipelines for searching for FRBs designed for use with HIRAX.  He
and collaborators tested the pipeline on archival Green Bank Telescope
(GBT) data, and in the process discovered one of the first FRBs known
\citep{Masui15}.  This FRB was the first one discovered with the GBT,
and the first one seen at CHIME/HIRAX frequencies.  Lower-frequency
searches had not seen any FRBs, and so this was the first event
clearly showing that CHIME/HIRAX would have things to see.  In
addition, careful analysis was able to show, although indirectly, that
this burst must have origin from outside the Milky Way.  It also
marked the first time that a burst was seen with linear polarization,
which showed that there was magnetized plasma near the source of the
FRB.  He is co-PI of HIRAX, and PI of the low-frequency work on
Marion.  He holds a Canada-150 Research Chair (\$1,000,000/year level)
and will fund the highly qualified personell required to built, test,
and deploy the \mbe\ from his C150 funds.  Sievers is an expert in
data analysis, and has the skills to handle and to train others to
handle the flood of data that the \mbe\ will produce, and turn it into
productive science.

Co-I Chiang has a long history of building and deploying instruments.
After an early start in particle physics, she has worked in Antarctica
on three separate experiments, BICEP and SPT from the South Pole, and
the balloon-borne SPIDER from the coastal McMurdo station.  She has
also worked on data from the Planck Satellite, the team that won the
Gruber Prize in cosmology for 2018.  She has co-authored well over 160
scientific publications, with over 30,000 citations (NASA ADS) and an
$h$-index of 76.  She is hardware lead for HIRAX, and has overseen the
successful construction HartRAO prototye, and development of the HIRAX
custom dishes.  She has additionally led the construction and
deployment of low-frequency antennas to Marion Island; this work is
particularly challenging since access to the site is via ship once per
year, for just three weeks.  Even a single missing or broken part
(without a spare) can delay progress for a year.  Chiang has the
expertise to install the \mbe, and to lead the exploration of northern
Quebec for good sites for future \mbe\ deployments.

The broader HIRAX team has the strengths required for a successful
experiment, with participations from around 25 institutions around the
globe\footnote{Collaborating institutions available at
  https://www.acru.ukzn.ac.za/~hirax/index.php/team/}.  Most
critically for the \mbe, McGill Professor Matt Dobbs, who developed
the FPGA boards at the heart of the \mbe, is a HIRAX collaborator and
will supply the expertise needed to build the \mbe.  Other key
Canadian collaborators include Keith Vanderlinde (Toronto) who build
the CHIME GPU-engine, and Kendrick Smith (Perimeter Institute) who
enhanced the algorithms in \citet{Masui15} and developed the FRB
search pipeline successfully used by CHIME. Gordon Lacy (NRC) has
demonstrated success with half-scale prototype dishes using
inexpensive materials, and shown the half-scale prototype surface
accuracy is a third of a millimetre.  Tzu-Ching Chiang (JPL) led
the first successful intensity mapping detection \citep{Chang10} and
her group contains experts in electromagnetic simulations modelling
HIRAX hardware.  Jeff Peterson (Carnegie-Melon) and Kevin Bandura
(West Virginia University) have long experience with radio-frequency
components and lead the RF development, and also worked on the first
intesnity mapping detection.  Laura Newburgh (Yale) brings a wealth of
experience from work on CHIME, and has played a major role in HIRAX
deployments to-date.  Aris Karastergiou (Oxford) will lead the pulsar
search with HIRAX, and Amanda Weltman (University of Cape Town) will
interpret the dark energy constraints that come from HIRAX.  Kavilan
Moodley (UKZN) took over as South African HIRAX PI after Sievers
joined McGill, and oversees coordination with SARAO and site
preparations.  Together, we have already fielded a working prototype,
and team members have expertise in all the key areas required to make
HIRAX work.  With significant funding in place, with contingency
backstopped by the C150 chair, the team has the skill and experience
to make the \mbe\ and HIRAX work, and push our understanding of the
universe forward.

\section{Infrastructure}
The mobile nature of the \mbe\ is unique in that it will be the only
truly mobile world-class back-end.  With the quick pace of GPU
developments, we expect the \mbe\ will be paired with many different
GPU engines over its life.  The \mbe\ is an essential part of HIRAX.
Without it, there will be no HIRAX, and Canada will miss a golden
opportunity to strengthen its staning in understanding dark energy and
FRBs.  By number of input signals, the \mbe\ will be the
second-largest F-engine in the world, trailing only CHIME\footnote{the
  total number of signals coming from the ASKAP, the Australian SKA
  Pathfinder, is also larger, but they are split into several
  independent groups which are not processed together}.  HIRAX will
begin construction late this year, with the 512-element stage starting
late 2020, and so it is on the same timescale as the \mbe.  Crucially,
there will be an array of radio telescopes to feed the \mbe.  While
the full HIRAX array is not yet funded, we are actively proposing, and
are confident we will be able to use the full capacity of the \mbe. 
%xxx

Since the hardware will be essentially identical to the F-engine build
by Matt Dobbs for CHIME, we have a very good indication of its cost.
The ICE boards, digitizers, and custom backplanes, with spares, will
cost \$729,600 USD, at a conservative exchange rate of 0.7, with an
effective tax rate of 5\%, plus the cost of racks/power supplies to
run the \mbe, totals for \$1,112,000 CAD.  The RF-shielded enclosure
(\$72,000 CAD), air-conditioned shipping container, and shipping
to/from the South African HIRAX site will be \$153,000 CAD (including
tax on the Canadian-acquired components).  Fortunately, for equipment
loaned to South African universities for research purposes that will
not stay in the country, there will be no South African duty or VAT
charged on the \mbe.  The power in South Africa from the mains is not
clean, with frequent voltage sags, and power losses.  To keep the \mbe
(along with the rest of HIRAX) working, we will need a diesel rotary
uninterruptable power supply (DRUPS).  SARAO has investigated the
matter carefully, and found these to be the most cost-effective
solution given the quality of the electricity supply.  That cost
will be about \$376,000 CAD, with UKZN purchasing this as an in-kind
contribution and McGill to retain ownership at the completion of
HIRAX.  We note that the DRUPS is overpowered for just the needs of the
\mbe\, but future deployments will also need to power the radio
receivers and the X-engine, and so the full power of the DRUPS will be used.

Specialized lab and field RF test equipment will be \$214,000 CAD
including tax, and specialized lab RF components including assortments
of amplifiers, filters, bias tees, FPGA boards etc. will be \$42,000.
Based on experience building stand-along antennas on Marion, the cost
per site-testing antennas is \$22,000 USD, split roughly evenly
between mini shipping containers to move the equipment (and
environmentally protect the electronics once on site), FPGA boards to
process the data, batteries to work off-grid, solar panels and
RF-quiet charge controllers, the antennas and associated RF hardware
themselves, and miscellaneous computers/drives/custom RF-tight
enclosures.  With exchange rate and tax, this works out to \$33,000
CAD per station, and we intend to deploy 3 stations to test a variety
of sites long enough to truly understand their RFI environment.  This
gives a total budget of \$99,000 for the base stations.  The total
request, including tax and matching contributions, is then \$1,996,000
CAD.  Matching will come from vendors (RF equipment supplier Keysight
has agreed to CFI matching) and from South African contributions.  The
test equipment will stay housed at McGill, modulo portable units
brought into the field for short tests.

\section{Institutional Commitment and Sustainability}

McGill University has made a significant and sustained commitment to
radio astronomy.  It has been a major part of the successful CFI
proposal that built CHIME, and led the CFI proposal that added the
capability to search CHIME data for fast radio bursts.  Without the
extensive previous commitments from McGill, supported through Quebec
and national funding, construction of the \mbe\ described here would
not be possible.  It would require years of research and development,
costing far more.
% More broadly, Canada's
%investment in CHIME is what has enable HIRAX, which relies on several
%key CHIME technologies.  
McGill has further invested in this area
through the hiring of Adrian Liu, a theorist working in low-frequency
radio astronomy.  Indeed, the institutional expertise in this area was
a major factor in attracting the proposers to McGill.

The South African Radio Astronomy Observatory has agreed to host the
core HIRAX site, including the \mbe.  As part of the site agreement,
SARAO has committed to providing access to power and data connections
for HIRAX.  The transmission line supplying power to HIRAX also
supplies the SKA site, and is near capacity, so the SARAO commitment
to provide access to power is essential.  The University of
KwaZulu-Natal (UKZN) has committed to site preparations and
backstopping the power installation costs, which will include power
and data line extensions and a transformer in addition to the DRUPS.
UKZN has further agreed to pay the power bill for HIRAX, including the
\mbe, during operations in South Africa.  The funding already secured
for HIRAX is enough to build a 128-element array that will run through
the \mbe.

With no moving parts, the reliability of the \mbe\ will be quite high
as long as the power supply is clean (hence the need for the DRUPS).
For the HIRAX deployment covered here, the \mbe\ will have a single
role, simplifying operations.  We have run the ICE board in the HIRAX
prototype for many months without needing human intervention, other
than rebooting after power outages.  After integration and
commissioning, the \mbe\ should run in a similarly hands-off manner.
However, given the scale of the the infrastructure, we will use
operating funds for part of an engineer's time to monitor the \mbe,
and troubleshoot as-needed.  This will amount to \$32,000/year over
five years.  In the event of a board failure (known to
be very rare), lab boards can be swapped in until replacement boards
can be acquired.  As a holder of a Canada 150 chair, Sievers will
cover the personnel costs associated with building the \mbe, and any
unforseen operational costs that may arise.

Chiang has already taken up the space at McGill for the radio lab.
The requested lab infrastructure has no special needs beyond basic
infrastructure like lab benches and power, and no renovations of the
current lab will be needed.  The antennas systems to be developed are
easily portable, and can be deployed from a pickup truck/large
SUV-class vehicle.  Operations for these will come from her NSERC
Discovery Grant (with northern supplement) for which she has applied.
While the most RFI-sensitive science cases may require truly remote
locations in the far north (such as at the McGill Arctic Research
Station on Axel Heiberg Island), the logistical challenges of larger
scale deployments of the \mbe\ make road-accessible regions of Quebec
the most natural place to search.  Costs associated with test antenna
deployments in road-accessible regions to identify promising future
sites will be modest, and Sievers will cover any further expenses
required for these test deployments from C150 funds.


\section{Benefits to Canada}

The infrastructure requested in this proposal directly addresses
several priorities and recommendations in the Canadian 2010 Long Range
Plan (LRP) for astronomy~\citep{lrp}.  Dark energy is highlighted
numerous times in the executive summary, underscoring the extreme
importance of this topic in modern cosmology.  The \mbe\ will expand
CHIME-like coverage to nearly the full sky, while enabling future
SKA-type science.  Support for CHIME and SKA science are both
emphsized in the Canadian astronomy plan.  Recommendation~7 of the LRP
panel ``reiterates the need for \ldots experimental astrophysics
laboratories for \ldots instrumentation and technology development.''
The \mbe\ will be designed, built, integrated, and tested at McGill.
While its first deployment will be overseas, the installation of the
\mbe\ will cement McGill's position as a leading institution within
HIRAX.  The construction of the radio lab at McGill by Cynthia Chiang
will further strengthen the Canadian and Quebec radio astronomy
profile.  In addition to the training of HQP, most of the equipment
for the lab will be acquired through Canadian suppliers.  Prototypes
will also be manufactured in Canada, such as the HIRAX composite dish
prototypes that have already been designed, built, and characterized
at NRC.  We stress that Canadian team members, both at McGill and
elsewhere, and at student, postdoc, and faculty level, will have full
access to all HIRAX data should the \mbe\ be funded.

% Canadian team members will have full access to the data, and
% will be expected to participate in all aspects of the data analysis
% and science outputs.  This includes members at student, postdoc, and
% faculty levels.

The LRP further states that ``Demonstrating and maintaining Canadian
sovereignty over its Arctic territory has become an important national
priority. One method of accomplishing this goal is through Arctic
research programs; a major initiative is now underway to exploit the
Canadian High Arctic for astronomical observations.''  The report
focuses on optical observations, but the infrastructure in this
proposal brings new added value by expanding the astronomy landscape
in northern Canada to include radio observations.  A major component
of this proposal is to build hardware to search for sites with very
low amounts of radio-frequency interference (RFI) for the next
deployment of the \mbe.  The ideal location for a large array is one
with access to power and internet, but far from population centers.
Northern Quebec is especially attractive for this due to the
infrastructure accompanying the region's extensive hydropower
generation.  Multiple fiber optic networks run north, operated by the
Eeyou Communications Network.  One line runs along the James Bay road
with several branches cutting west to service towns along the coast.
A second line runs to the east, passing near Nemaska.  This level of
infrastructure in some of the least densely populated areas in the
planet is a truly unique opportunity.  With the revolution in radio
astronomy enabled by infrastructure like the \mbe, the shift towards
inexpensive dishes means we can hire local labour to build and install
future arrays.  This model has already worked well in South Africa,
where roughly twenty workers with no previous experience have been
hired to build the Hydrogen Epoch of Reionization Array (HERA),
another radio array hosted by SARAO.

In addition to addressing national priorities, the proposed
infrastructure also aligns strongly with the Qu\'ebec Research and
Innovation Strategy (SQRI).  One of the top priorities of the SQRI is
analytics of massive datasets.  The \mbe\ will process over 50 {\it
  exabytes} of digital data during its first deployment.  This data
set will be averaged to a daily rate of $\sim$50 TB/day, or over 70~PB
during the life of HIRAX.  In order to manage this enormous amount of
data, our team will develop novel compression techniques without
compromising the fidelity of the cosmology results, as well as write
algorithms to search for unexpected signals.  Due to the massive
volume, compression for HIRAX will have to happen on-site; with
redundancy built into the array the compressed volume should be a few
hundred GB per day.  We will copy this data back to Canada where we
will analyze the full survey on Compute Canada/Calcul Qu\'ebec machines.
Detailed instrument modelling and simulations are key to the success
of HIRAX and interpretation of the results; this software development
addresses the SQRI goal of ``modelling, simulation, and games.''  The
\mbe\ will additionally open new research paths in artificial
intelligence, another goal that is specifically highlighted by the
SQRI.  Artificial intelligence has already been used to make
significant progress in FRB searches, and this initial demonstration
represents only the beginning of the full power that artificial
intelligence techniques have to offer.


\subsection{Highly Qualified Personel}


The infrastructure that is supported by this proposal will provide a
unique and interdisciplinary training environment for HQP.  The HQP
will have opportunities to play key roles in the detailed design of
the \mbe, subsystem testing, end-to-end integration, fielding of the
\mbe, and design/construction of new radio telescope stations that
will be supported by the \mbe.  The range of work is diverse and will
encompass both science and engineering.  In order to achieve the above
research goals, HQP will will be trained in mechanical and
electromagnetic design, electronics design and operation, instrument
characterization techniques, system integration, field work and
observations, and data analysis.  This type of training, in addition
to experience in leadership and mentoring, is critical for the next
generation of young scientists to push fundamental research forward.
The skills listed above are highly transferable and sought after by
industry, and the proposed infrastructure will thus open a wide range
of career options for HQP to contribute the fruits of their training
to positive socioeconomic growth.  A highly trained workforce is one
of the primary deliverables from the mission driven innovation
supported by this proposal.  Since the MBE will be installed in
multiple locations, the proposed infrastucture will involve a strong
component of international collaboration.  Interactions between HQP in
multiple countries will foster the growth of a racially- and
gender-diverse cohort.  PI and co-PI Sievers and Chiang already have
an exceptionally strong track record in bringing together HQP from a
wide range of backgrounds.  Since starting as permanent faculty in
2013, they have trained a combined total of 26 HQP representing eleven
different countries (Botswana, Chile, DRC, Ethiopia, Ghana, India,
Kenya, South Africa, Turkey, USA, Chile), including several students
from historically disadvantaged backgrounds.  The acquisition of the
\mbe\ will enable continued training of HQP with each PI supervising
5-6 HQP on average during the duration of the project through direct
supervision, for a total of 20-30 HQP.  Many more HQP will train at
other institutions in Canada and across the world.
%Past and
%present HQP trained by Sievers and Chiang since 2013 include
%* 1 research associate: R. Monsalve \\
%* 7 postdocs: M. Aich, B. Saliwanchik, H. Heilgendorff, T. Voytek, K. Knowles, S. Muya-Kasanda, V. Singh \\
%* 5 PhDs: T. Gogo, H. Heilgendorff, D. Olcek, L. Philip, O. Sengate \\
%* 8 MScs: J. Allotey, N. Ghazi, A. Gumba, K. Kesebonye, B. Kushiator, M. Mdlalose, O. Sengate, A. Zungu \\
%* 10 BSc Hons: N. Ghazi, A. Gumba, M. Gumede, K. Kesebonye, T. Moso, S. Ngobese, O. Seremane, W. Roberson, A. Moola, M. Mngqibisa \\
%
%Repeated names in the above list indicate successful completion of
%degrees and continued progress in academic studies.  Career milestones
%listed below include permanent jobs, prize fellowships, and other
%notable achievements of HQP trained by either Sievers or Chiang.  \\
%* V. Singh: Permanent research faculty at the Physical Research
%Laboratory\\
%* A. Zungu: Lecturer position at Sol Plaatje University\\
%* R. Monsalve: Lead postdoc on EDGES, and passed on many other
%opportunities to work on \mbe-related work at McGill.\\
%* M. Aich: Research scientist post, UKZN\\
%* K. Knowles: Square Kilometre Arrray postdoctoral fellowship\\
%* H. Heilgendorff: Square Kilometre Array postdoctoral fellowship\\
%* B. Saliwanchik: postdoctral position at Yale\\
%* O. Sengate: will be first-ever Botswanan astronomy PhD\\
%* J. Allotey: Square Kilometre Array internship, followed by PhD studies at Bristol\\
%* Nearly all other students in the above list have continued on to the
%next stages of their postgraduate studies\\


%* In early 2018, six students applied for the Dunlap/UToronto
%  astronomical instrumentation summer school. Five were accepted,
%  demonstrating their excellence among international standards. The
 % students have also participated in numerous other conferences and
 % workshops.  [ANYTHING NOTEWORTHY TO MENTION HERE?]
%

%Publications involving HQP:%
%
%* L. Philip et al., ``Probing Radio Intensity at high-Z from Marion:%
% 2017 Instrument,'' submitted to JAI 2018. L. Philip is the lead
% author and is my PhD student.
%* M. Jones et al., ``The C-Band All-Sky Survey (C-BASS): design and
% capabilities'', MNRAS 480, 2018. My postdoc and former PhD student,
% H. Heilgendorff, played a key role in developing the data analysis
% methods.
%* L. Newburgh et al., ``HIRAX: A Probe of Dark Energy and Radio
% Transients'', Proc SPIE 9906, 2016. One postdoc and two students
% (B. Saliwanchik, A. Gumba, L. Ngwenya) made critical contributions
% to instrument construction.

%[POSSIBLY USEFUL TEXT COPIED BELOW, NEEDS MORE WORK]

%These opportunities include interactions with industrial partners that
%are involved with subsystem manufacturing for our projects. The HQP
%who we have trained have backgrounds including physics, astronomy,
%mathematics, and engineering, and this broad spectrum yields many
%fruitful interactions.




%be very clear about added value to Quebec
%make sure to have sustainability part
%can cut down on intro
%make clear that due to fiber constraints the back end needs to be next to the dishes
%make more clear who HIRAX collaboration is, and who ``we'' refers to, e.g. HIRAX, Cynthia and me, Canada in general
%be careful to be clear why we need this tool as a JELF, not as part of a collaboration.  Always come back to own research goals.
%make clear outriggers are not part of MBE.  ``everything is interconnected, but always come back to what 
%   we are going to do with infrastructure here.''
%be clear that timescale to get MBE working is same as timescale to get HIRAX big enough to be used.  
%   this should go in the need for infrastructure.  be clear that we are still looking, and that 
%   we can use MBE with the things we already have funded.  
%get an email from Jo-Ann.  say that we have secured space from parking at McGill for integration here.  Make sure to
%   mention that containers will be locked/secured.  MAKE SURE TO GET 
%email site letter to Nathalie/Ada.  need to discuss operation/maintenance.  make clear who is paying for what.  
%   add that power will be covered by South African side in sustainability.  
%radio lab for building antennas.  make clear that will be Cynthia's lab at McGill, equipment will stay here, and we need it to
%    develop the infrastructure.
%describe in researchers section to be clear that we personally have tools/skills to use equipment here.  
%    JELF needs to be clear exactly what the role in development, but also in exploitation.  
%    needs to be clearly seen that we have expertise to exploit.  Matt is essential collaborator that we need
%    to develop the use of tools for our own research programme.  ``build on expertise Matt has
%    to go to next step on data where we have the expertise to exploit the data.''
%explain HIRAX consortium more strongly.  But, make clear we can't continue our research without access
%    to this piece of equipment.  Benefits to Canada and Quebec should make clear that science focus observations
%    move strongly to Canada.  
%McGill have already built something around radio, e.g. CHIME pathfinder, CHIME, demonstrates institutional commitment, and we have
%    network of collaborators here in place.  Look at CHIME CFIs, Vicki has been on a third one.  We can make clear
%    that McGill has made a serious commitment to this stuff, and that we can build on what has been funded.
%    make very clear that HIRAX has been enabled by previous Canadian investments in CHIME, put in sustainability.
%Use 8% for tax rate.  
%Be more clear about what is included in DRUPS cost.  Power is operating and doesn't count as in-kind.  
%    cost to extend power line to DRUPS, data connection are CFI-countable in kind probably.  
%    write in infrastructure section that installation costs will be born by UKZN.  
%    add installation costs to make sure line item is large enough.  UKZN should be vendor so 
%    they show up as in-kind.  UKZN should keep track of any costs/invoices associated with installation.  
%UKZN Administration needs to co-sign eventually the matching letter.  Nathalie needs to go through funding, site operations, maintenance. 
%Ada will share blurbs, need to talk about operations and maintenance.  
%IOF - 30% of 40% CFI, so 12% of CFI, we get 2/3, so 8%= 160K to use over 5 year period.  Should go in text in sustainability.  
%do not call parts spare parts, make clear they are for lab testing.  ``ICE Board Starter Kit''
%    Letter from Matt is fine for quote.  
%Don't talk about second container since coming from elsewhere.  Don't use word ``container'' for GPUs.  
%Ada will share with us quebec strategy, so we can align.  


\bibliographystyle{apj}
\bibliography{cfi_jelf_sievers}{}


%total - 1112+153+376+214+42+33*5

\end{document}
